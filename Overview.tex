%%% Local Variables:
%%% mode: latex
%%% TeX-master: t
%%% End:

\documentclass{article}
\usepackage[1in]{fullpage}
\usepackage{hyperref}
\usepackage[utf8]{inputenc}
\usepackage[english]{babel}
\usepackage{enumitem}

\usepackage{minted}
\title{Maybe Overview}
\author{Yihong Chen}
\date{}
\begin{document}

\maketitle

Maybe project let developers be able to defer choices about app behavior that difficult to made at development time.
\section{Android Library}
\label{sec:android-library}
\begin{enumerate}[label=\alph*.]
\item Communicate with \href{https://developer.android.com/google/gcm/index.html}{Google Cloud Messaging} and maybe backend server.

Why Google Cloud Messaging? When backend server get a new choice(s) for one device, we need \textbf{push} the change to device.
\item Collect logs from device.

It should be able to cache logs and do batch upload.
\item TODO: get decision tree from backend server and able to dynamically make decision on device.
\end{enumerate}

\section{Backend Server}
\label{sec:backend-server}

\subsection{Requirements}
\label{sec:requirements}

\begin{enumerate}[label=\alph*.]
\item provide choices for all devices (each device may has different choices)
\item push changes of choice
\item collect logs
\item TODO: send decision tree to device
\end{enumerate}

\subsection{Tech roadmap}
\label{sec:tech-roadmap}
We use \href{https://www.meteor.com/}{Meteor} to develop backend.

repo address: \url{https://github.com/xcv58/backend}

\section{Rewriter}
\label{sec:rewriter}
Change maybe style codes to ordinary Java code. Such as:
convert
\begin{minted}{java}
maybe ("label") {
    // do something 0
} or {
    // do something 1
} or {
    // do something 2
}

\end{minted}
to
\begin{minted}{java}
int choice = maybeService.get("label");
switch (choice) {
    case 0:
        // do something 0
        break;
    case 1:
        // do something 1
        break;
    case 2:
        // do something 2
        break;
    default:
        maybeService.log("get invalid choice for " + choice + " label");
        break;
}
\end{minted}
TODO: IDE support
TODO: build system integration

\section{Learning Part}
\label{sec:learning-part}
\end{document}
