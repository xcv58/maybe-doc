%%% Local Variables:
%%% mode: latex
%%% TeX-master: t
%%% End:

\documentclass{article}
\usepackage[1in]{fullpage}
\usepackage{hyperref}
\usepackage[utf8]{inputenc}
\usepackage[english]{babel}
\usepackage{enumitem}
\usepackage{minted}
\hypersetup{%
  colorlinks=true,% hyperlinks will be coloured
  linkcolor=green,% hyperlink text will be green
  linkbordercolor=red,% hyperlink border will be red
}

\newcommand{\javafile}[1]{\inputminted
[frame=lines,
framesep=2mm,
baselinestretch=1.2,
fontsize=\footnotesize,
linenos]
{java}
{#1}}

\title{Maybe Overview}
\author{Yihong Chen}
\date{}
\begin{document}

\maketitle

Maybe project let developers be able to defer choices about app behavior that difficult to made at development time.
\section{Android Library}
\label{sec:android-library}
\begin{enumerate}[label=\alph*.]
\item Communicate with \href{https://developer.android.com/google/gcm/index.html}{Google Cloud Messaging} and maybe backend server.

Why Google Cloud Messaging? When backend server get a new choice(s) for one device, we need \textbf{push} the change to device.
\item TODO: Collect logs from device.

It should be able to cache logs and do batch upload.
\item TODO: get decision tree from backend server and able to dynamically make decision on device.

\item TODO: debug mode.

Let develop easily attach a setting activity for maybe variables. It should be generated by meta data.
\end{enumerate}

\section{Backend Server}
\label{sec:backend-server}

\subsection{Requirements}
\label{sec:requirements}

\begin{enumerate}[label=\alph*.]
\item provide choices for all devices (each device may has different choices)

different schemas for value assignments of new device, such as: random,
fixed, preassigned
\item push changes of choice
\item TODO: collect logs, \href{http://docs.mongodb.org/ecosystem/use-cases/storing-log-data/}{Storing Log Data in mongoDB}
\item TODO: send decision tree to device
\item TODO: user system and permission control
\end{enumerate}

\subsection{Tech roadmap}
\label{sec:tech-roadmap}
We use \href{https://www.meteor.com/}{Meteor} to develop backend.

repo address: \url{https://github.com/xcv58/backend}

\section{Rewriter}
\label{sec:rewriter}
Change maybe style codes to ordinary Java code. Such as:
convert

\javafile{maybe-origin.java}

to

\javafile{maybe-rewrite.java}

\subsection{Current Implementation}
\label{sec:curr-impl}
Current version uses Python to implement rewriter. The basic idea is using
regular expression to match maybe syntax, and convert it to switch case
code block. In the meanwhile, it will upload metadata to maybe backend.
repo: \url{ssh://src@blue.cse.buffalo.edu/projects/maybe/rewriter}

But we need change some feature of current implemenatation.

\subsection{TODO: Rewrite whole project instead one Java file}
\label{sec:rewr-whole-proj}
Current implemenatation can only rewrite one Java file. It takes the
Java package name as the package name used in maybe metadata.

But we need an unique namespace for an Android project.
The Android package name is unique for an Android project.
More details about Android package name can be found:
\url{http://developer.android.com/guide/topics/manifest/manifest-element.html}

\subsection{TODO: IDE Support}
\label{sec:ide-support}
The maybe syntax should be supported by IDE like Eclipse and Android Studio
(IntelliJ IDEA). For Android Studio, we need to develop a plugin with
grammar support for the syntax detection.

TODO: how to integrate other features of IDE like auto complete.

\subsection{TODO: Build System Integration}
\label{sec:build-syst-integr}
For build system, we can integrate our rewriter with Gradle.
This highly depends how we implement new version of rewriter.

\section{TODO: Learning Part}
\label{sec:learning-part}

\end{document}
